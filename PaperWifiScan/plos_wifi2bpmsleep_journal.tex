% Template for PLoS
% Version 3.5 March 2018
%
% % % % % % % % % % % % % % % % % % % % % %
%
% -- IMPORTANT NOTE
%
% This template contains comments intended 
% to minimize problems and delays during our production 
% process. Please follow the template instructions
% whenever possible.
%
% % % % % % % % % % % % % % % % % % % % % % % 
%
% Once your paper is accepted for publication, 
% PLEASE REMOVE ALL TRACKED CHANGES in this file 
% and leave only the final text of your manuscript. 
% PLOS recommends the use of latexdiff to track changes during review, as this will help to maintain a clean tex file.
% Visit https://www.ctan.org/pkg/latexdiff?lang=en for info or contact us at latex@plos.org.
%
%
% There are no restrictions on package use within the LaTeX files except that 
% no packages listed in the template may be deleted.
%
% Please do not include colors or graphics in the text.
%
% The manuscript LaTeX source should be contained within a single file (do not use \input, \externaldocument, or similar commands).
%
% % % % % % % % % % % % % % % % % % % % % % %
%
% -- FIGURES AND TABLES
%
% Please include tables/figure captions directly after the paragraph where they are first cited in the text.
%
% DO NOT INCLUDE GRAPHICS IN YOUR MANUSCRIPT
% - Figures should be uploaded separately from your manuscript file. 
% - Figures generated using LaTeX should be extracted and removed from the PDF before submission. 
% - Figures containing multiple panels/subfigures must be combined into one image file before submission.
% For figure citations, please use "Fig" instead of "Figure".
% See http://journals.plos.org/plosone/s/figures for PLOS figure guidelines.
%
% Tables should be cell-based and may not contain:
% - spacing/line breaks within cells to alter layout or alignment
% - do not nest tabular environments (no tabular environments within tabular environments)
% - no graphics or colored text (cell background color/shading OK)
% See http://journals.plos.org/plosone/s/tables for table guidelines.
%
% For tables that exceed the width of the text column, use the adjustwidth environment as illustrated in the example table in text below.
%
% % % % % % % % % % % % % % % % % % % % % % % %
%
% -- EQUATIONS, MATH SYMBOLS, SUBSCRIPTS, AND SUPERSCRIPTS
%
% IMPORTANT
% Below are a few tips to help format your equations and other special characters according to our specifications. For more tips to help reduce the possibility of formatting errors during conversion, please see our LaTeX guidelines at http://journals.plos.org/plosone/s/latex
%
% For inline equations, please be sure to include all portions of an equation in the math environment.  For example, x$^2$ is incorrect; this should be formatted as $x^2$ (or $\mathrm{x}^2$ if the romanized font is desired).
%
% Do not include text that is not math in the math environment. For example, CO2 should be written as CO\textsubscript{2} instead of CO$_2$.
%
% Please add line breaks to long display equations when possible in order to fit size of the column. 
%
% For inline equations, please do not include punctuation (commas, etc) within the math environment unless this is part of the equation.
%
% When adding superscript or subscripts outside of brackets/braces, please group using {}.  For example, change "[U(D,E,\gamma)]^2" to "{[U(D,E,\gamma)]}^2". 
%
% Do not use \cal for caligraphic font.  Instead, use \mathcal{}
%
% % % % % % % % % % % % % % % % % % % % % % % % 
%
% Please contact latex@plos.org with any questions.
%
% % % % % % % % % % % % % % % % % % % % % % % %

\documentclass[10pt,letterpaper]{article}
\usepackage[top=0.85in,left=2.75in,footskip=0.75in]{geometry}

% amsmath and amssymb packages, useful for mathematical formulas and symbols
\usepackage{amsmath,amssymb}

\DeclareMathOperator{\atantwo}{atan2}
\DeclareMathOperator{\arctantwo}{arctan2}
\DeclareMathOperator*{\argmax}{argmax} % thin space, limits underneath in displays
\usepackage{diagbox}

% Use adjustwidth environment to exceed column width (see example table in text)
\usepackage{changepage}

% Use Unicode characters when possible
\usepackage[utf8x]{inputenc}

% textcomp package and marvosym package for additional characters
\usepackage{textcomp,marvosym}

% cite package, to clean up citations in the main text. Do not remove.
\usepackage{cite}

% Use nameref to cite supporting information files (see Supporting Information section for more info)
\usepackage{nameref,hyperref}

% line numbers
\usepackage[right]{lineno}

% ligatures disabled
\usepackage{microtype}
\DisableLigatures[f]{encoding = *, family = * }

% color can be used to apply background shading to table cells only
\usepackage[table]{xcolor}

% array package and thick rules for tables
\usepackage{array}

% create "+" rule type for thick vertical lines
\newcolumntype{+}{!{\vrule width 2pt}}

% create \thickcline for thick horizontal lines of variable length
\newlength\savedwidth
\newcommand\thickcline[1]{%
	\noalign{\global\savedwidth\arrayrulewidth\global\arrayrulewidth 2pt}%
	\cline{#1}%
	\noalign{\vskip\arrayrulewidth}%
	\noalign{\global\arrayrulewidth\savedwidth}%
}

% \thickhline command for thick horizontal lines that span the table
\newcommand\thickhline{\noalign{\global\savedwidth\arrayrulewidth\global\arrayrulewidth 2pt}%
	\hline
	\noalign{\global\arrayrulewidth\savedwidth}}


% Remove comment for double spacing
%\usepackage{setspace} 
%\doublespacing

% Text layout
\raggedright
\setlength{\parindent}{0.5cm}
\textwidth 5.25in 
\textheight 8.75in

% Bold the 'Figure #' in the caption and separate it from the title/caption with a period
% Captions will be left justified
\usepackage[aboveskip=1pt,labelfont=bf,labelsep=period,justification=raggedright,singlelinecheck=off]{caption}
\renewcommand{\figurename}{Fig}

% Use the PLoS provided BiBTeX style
\bibliographystyle{plos2015}

% Remove brackets from numbering in List of References
\makeatletter
\renewcommand{\@biblabel}[1]{\quad#1.}
\makeatother



% Header and Footer with logo
\usepackage{lastpage,fancyhdr,graphicx}
\usepackage{epstopdf}
%\pagestyle{myheadings}
\pagestyle{fancy}
\fancyhf{}
%\setlength{\headheight}{27.023pt}
%\lhead{\includegraphics[width=2.0in]{PLOS-submission.eps}}
\rfoot{\thepage/\pageref{LastPage}}
\renewcommand{\headrulewidth}{0pt}
\renewcommand{\footrule}{\hrule height 2pt \vspace{2mm}}
\fancyheadoffset[L]{2.25in}
\fancyfootoffset[L]{2.25in}
\lfoot{\today}

%% Include all macros below

\newcommand{\lorem}{{\bf LOREM}}
\newcommand{\ipsum}{{\bf IPSUM}}

%% END MACROS SECTION


\begin{document}
	\vspace*{0.2in}
	
	% Title must be 250 characters or less.
	\begin{flushleft}
		{\Large
			\textbf\newline{Sleep Respiratory Rate Monitoring with Portable Wi-Fi} % Please use "sentence case" for title and headings (capitalize only the first word in a title (or heading), the first word in a subtitle (or subheading), and any proper nouns).
		}
		\newline
		% Insert author names, affiliations and corresponding author email (do not include titles, positions, or degrees).
		\\
		Name1 Surname\textsuperscript{1,2\Yinyang},
		Name2 Surname\textsuperscript{2\Yinyang},
		Name3 Surname\textsuperscript{2,3\textcurrency},
		Name4 Surname\textsuperscript{2},
		Name5 Surname\textsuperscript{2\ddag},
		Name6 Surname\textsuperscript{2\ddag},
		Name7 Surname\textsuperscript{1,2,3*},
		with the Lorem Ipsum Consortium\textsuperscript{\textpilcrow}
		\\
		\bigskip
		\textbf{1} Affiliation Dept/Program/Center, Institution Name, City, State, Country
		\\
		\textbf{2} Affiliation Dept/Program/Center, Institution Name, City, State, Country
		\\
		\textbf{3} Affiliation Dept/Program/Center, Institution Name, City, State, Country
		\\
		\bigskip
		
		% Insert additional author notes using the symbols described below. Insert symbol callouts after author names as necessary.
		% 
		% Remove or comment out the author notes below if they aren't used.
		%
		% Primary Equal Contribution Note
		\iffalse
		\Yinyang These authors contributed equally to this work.
		
		% Additional Equal Contribution Note
		% Also use this double-dagger symbol for special authorship notes, such as senior authorship.
		\ddag These authors also contributed equally to this work.
		
		% Current address notes
		\textcurrency Current Address: Dept/Program/Center, Institution Name, City, State, Country % change symbol to "\textcurrency a" if more than one current address note
		% \textcurrency b Insert second current address 
		% \textcurrency c Insert third current address
		
		% Deceased author note
		\dag Deceased
		
		% Group/Consortium Author Note
		\textpilcrow Membership list can be found in the Acknowledgments section.
		\fi
		% Use the asterisk to denote corresponding authorship and provide email address in note below.
		* m6222040385@g.siit.tu.ac.th
		
	\end{flushleft}
	% Please keep the abstract below 300 words
	\section*{Abstract}
	
	\iffalse

	WiFi human sensing is being challenged by researchers continuously e.g. Activity Classification, Localization and etc. However, we believed WiFi can do more. In this paper, we paired WiFi transmitting pattern to human sleep stage to create a mapping rule of those. we used ESP32 which is a reachable single board computer and strengthened its WiFi signal with 2 directional external antennae. After gathering WiFi variation with Sleep Stage from a sleep tracker, we applied machine learning to merge them. We used Long-short term memory (LSTM) to the framework since we know pattern of moving matters to Sleep Stage. So, we are to map a sleeping stage to a sequence of WiFi snapshots instead of one-to-one like others which makes this called ``Sleep monitoring''. we tested with newly collected data sizing more than 6,000 frames of sequences. To evaluate the result, we compared annotated Sleep Stage with predicted Sleep Stage and summarized their accuracy which finally resulted as that it is possible to classify 4 human sleep stages from WiFi in ESP32.
	
		\fi
	
	\linenumbers
	
	% Use "Eq" instead of "Equation" for equation citations.
	\section*{Introduction}
	Wi-Fi is one of the most common network mediums nowadays. Pervasively, it is used for establishing a wireless network to connect to the internet. But, there are still many more functions Wi-Fi is good at. Wi-Fi can also be applied in fields beside connecting to the internet according to its stability being upgraded continuously. Decent Wi-Fi connectivity can extract more data other than the data to be transmitted like concentration, speed, obstacle between the transmission. Those can be composed to be many useful applications like Localization, Activity Classification and etc.
	
	In order to achieve the applications like mentioned, There are many works tried to extract deep features from Wi-Fi. But, they are mostly working with very specific tools and old Network Interface Card (NIC) connected to a laptop running Linux that is currently one of the ways allowing to obtain fine-grain Channel State Information (CSI), the descriptive data of the  Wi-Fi propagating in that environment. Those limitations significantly decrease simplicity of implementation. It is hard for public demonstration and integration with many updated tools.
	
	Actually, there are other existing ways for obtaining the CSI. One is from a ubiquitously used microprocessor, ESP32. which is still not much explored in Wi-Fi exploiting field. It is simple to implement and can be easily integrated with other tools in many platforms due to its massively produced external tools. 
	
	
	Human sleep monitoring is being observed for a whlie and recently applied on commercial products. Sleeping quality is significant to all persons and can be investigated to many symptoms e.g. Sleep Apnea, which is a very serious disease that is widely detected in the elderly. People are more concerned on observing their sleepness by using devices those are able to inform them their sleep information e.g. smartwatches, smart mattress, smart rings and etc. The main feature that helps to detect the Sleep Apnea is a respiration monitoring because those patient of the disease is unable to breath constantly at night. So, the respiration monitoring can show consistency of breathing at night through a ``Respiratory Rate (RR)" which is in the unit of ``Breath Per Minute (bpm)''. But, those devices can considerably affect sleepness since it needs a contact to users' bodies. 
	
	So, this paper proposes a contactless sleep resipratory rate monitoring with Wi-Fi CSI from ESP32 where users can applied with easy-to-find microprocessor and monitor their respiration during sleep without contacting to their bodies.
	This allows sleep monitoring to be practical for other researchers and those who want to observe their health through sleepness.
	
	
	
	\section*{Background}
	
	\subsection*{Wi-Fi}\label{wifi}
	
	Wi-Fi is a well-known connectivity with no wire needed (wireless). It has been used as a medium for connecting to the internet for over 10 years. However, the Wi-Fi is the name covering IEEE 802.11 n/g/ac protocols. It delivers data through 2.4/5GHz frequency with multiple channels. The bandwidth in each channel is 22MHz. the data are to be transmitted  parallelly with multiplexing technique named orthogonal frequency division multiplexing (OFDM). Each carrier may propagate to a receiver with encountering many obstacles. The effect of that situation is the Doppler Effect.
	So, Channel State Information (CSI) is represented as physical layer indicator that can be used to investigate how each channel propagate to the receiver or back to the transmitter.
	
	If a sender sends data to a receiver through Wi-Fi, the data will be almostly not transmitted without any interference.
	
	
	
	\subsection*{CSI data}\label{CSI}
	As mentioned in \nameref{wifi} that data propagating to the receiver while touching surrounding environment, the CSI is a variation of the data. The CSI can  be found at both sender and receiver since receiver may transmit data back. Let the sender use the modulation method of 16-quadrature amplitude modulation (16-QAM) which one carrier can carry 4 bits. When the sender needs to send a `1111', the modulation returns $x=1+1i$. Then, transmit to the receiver. At the receiver, let the obtained data is $y=0.8+0.9i$. So, the CSI can be computed by the variation $h=y/x=0.2+3.4i$.
	
	Human body is literally water which reflect radio wave like Wi-Fi. \cite{wangF}, \cite{liuJ} and \cite{chowdhuryTZ} have proven that human body can affect the CSI.
	
	\subsubsection*{ESP32}\label{ESP32}
	ESP32 is a very popular single-board computer (SBC). With its affordable price and many available additional tools, ESP32 is commonly used in Internet of Things field.  Quantitative CSI can be obtained from Wi-Fi in ESP32 according to \cite{atifM}. The number of available subcarriers in ESP32 is 64.
	
	According to the detail about Wi-Fi mentioned in \nameref{wifi}, the Wi-Fi in ESP32 has some limitation. It supports only 2.4GHz frequency and can be set only one channel over a connection. The bandwidth of each channel is 22MHz. The CSI can be both obtain from Access point (AP) and Station (STA) as shown in Fig.~\ref{fig:ESP32CSI01}.  In this paper, we consider to mainly use CSI at the AP.
	The frequency of each channel is as 802.11 standard.
	
	
	\begin{figure}[htbp]
		\centerline{\includegraphics[width=85mm,scale=0.5]{ESP32.jpg}}
		\caption{2 ESP32s with external antenna. One act as a sender and another act as a receiver.}
		\label{fig:ESP32}
	\end{figure}
	
	
	\begin{figure}[htbp]
		
		\centerline{\includegraphics[width=70mm,scale=0.5]{ESP32CSI01.png}}
		\caption{CSI from ESP32s with channel 6.}
		\label{fig:ESP32CSI01}
	\end{figure}
	
	\iffalse
	\subsubsection*{Faraday cage}
	Faraday cage is invented by Michael Faraday in 1836. It is an enclosure to block electromagnetic fields. It is made of conducting material which can affect any radio frequency (e.g. Wi-Fi) to be unable to pass through.
	\fi
	
		\subsection*{Human sleep respiratory rate monitoring}\label{HUMANSLEEPRR}
	Human sleep respiratory rate monitoring is widely used and can help the investigation for identifying further incoming diseases as stated above. The common respiratory rate of human is various by age ranges and bodies (ref). But, it can be fluctuated during an abnormal situation.  However respiratory rate during sleep of a certain person should be consistant (ref). Otherwise, the person can be in danger of many serious diseases e.g. Bradypnea, Sleep Apnea and etc. 
	
	The simplest unit of respiratory rate is $bpm$, which tells how many time a certain person breaths in a minute. Many sensoring devices can log such information in time sequence. We use Vernier Respiration Monitor Belt, a device from the reliable sensoring manufacturer and commonly used in acadamic field, as our ground truth (ref). The belt is set to log RR at resolution of 4 Hz and The ESP32 is set to log CSI at resolution of 60 Hz. So, we herein create a mapping rule from a sequence of 60 CSI data to a sequence of 4 RR data for each second.
	
	\begin{figure}[htbp]
		\centerline{\includegraphics[width=85mm,scale=0.5]{VERNIERBELT.jpg}}
		\caption{Vernier Respiration Monitor Belt as ground truth.}
		\label{fig:VERNIERBELT}
	\end{figure}
	
	\section*{Materials and methods}
	
		\begin{figure}[htbp]
		\centerline{\includegraphics[width=40mm,scale=0.2]{SETUP05.jpg}}
		\caption{Experimental setup.}
		\label{fig:SETUP01}
	\end{figure}
	
	\begin{figure}[htbp]
		\centerline{\includegraphics[width=40mm,scale=0.2]{SETUP03.jpg}}
		\caption{Fitbit sleep tracker.}
		\label{fig:SETUP02}
	\end{figure}
	
	
	\subsection*{Concept}
	\label{concept}
	
	\begin{figure}[htbp]
		
		\centerline{\includegraphics[width=85mm,scale=0.5]{ESP32CSI03.png}}
		\caption{2 different CSIs resulted from corresponding human poses.}
		\label{fig:ESP32CSI02}
	\end{figure}
	
	Other famous proposed works for Activity Classification like \cite{wangF} \cite{liuJ} and \cite{hernandezSM} use omnidirectional antennae and focus at the line of sight between  the antennae. our work does likewise as shown in Fig.~\ref{fig:SETUP01}.
	
	
	The CSI is not only affected by human body but also by overall environment. This means that 2 identical human poses can result obviously different CSIs if the environment around are not exactly the same as shown in Fig.~\ref{fig:ESP32CSI02}. In short, CSI is suitable for moving target since we can focus on its change.
	
	 The example of mapping CSI to Activity Classification can be found in \cite{chowdhuryTZ} and \cite{zouH}. In the normal environment of sleeping, we assume that nothing in the room is moving besides an alive sleeping human on the bed. So, the corresponding changing of RR during sleep may affect to the same changing pattern of the CSI. This hypothesis is investigated in the upcoming parts.
	
	All steps of the mapping rule are shown in Fig.~\ref{fig:MAPINGSTEP}. 
	

	\begin{figure}[htbp]
		\centerline{\includegraphics[width=90mm,scale=0.2]{STEP12.png}}
		\caption{All the mapping rule.}
		\label{fig:MAPINGSTEP}
	\end{figure}
	
	\subsection*{Prediction (ESP32)}
	
	We applied (ref-steve-ESP32-CSI) to the ESP32 at receiver side to obtain real-time CSI. 
	
	However, original CSI data is complex number so, we can either convert it into phase or amplitude. As mentioned in \nameref{CSI}, the CSI is originally in form $h=y/x=v+wi$ so, they are parsed. For amplitude,  $\sqrt{ v^2+w^2 }$ and $\arctantwo(v, 2 )$ for phase.
	In this paper, we use amplitude.
	\begin{equation}
	\text{Amplitude} =  {  \sqrt{ v_{sc}^2+w_{sc}^2 } , sc \in [1, 52]}
	\label{eq:CSIampParser}
\end{equation}	
	There are 52 subcarrier in ESP32 (ref). We simply pick one.
	
	\subsubsection*{Hampel Filtering}
	
	The CSI amplitude parsed from raw CSI of ESP32 is originally noiseful and contain many outliers, a value which is significantly far from beside data. The CSI sequence is ideally a continuous data, so outliers are considered as noises.  Hampel Filtering is a filtering process to remove obvious outliers. An example of the result is shown in  Fig.~\ref{fig:HAMPEL} where the top is an input that is having many outliers and the bottom is an output that is having lower outliers.
	
		Parameters of the process are window size and $\sigma$ which are redefined as $\alpha$ and $\beta$ respectively in this paper.
		
		\begin{figure}[htbp]
		\centerline{\includegraphics[width=120mm,scale=0.9]{FILPD_R2H.png}}
		\caption{An example of applying Hampel filtering when $\alpha=1$ and $\beta=-100$.}
		\label{fig:HAMPEL}
	\end{figure}

	\subsubsection*{Gaussian Filtering}
	
	 The CSI after applying Hampel filtering is more clear but, still lag smoothness ostensibly. To smoothen the CSI, we used Gaussian filtering as (ref Activity Class).
	An example of the result is shown in  Fig.~\ref{fig:GAUSSIAN} where the top is an input that is very rough and the bottom is an output that is smoother.
	
	A parameter of the process is $\sigma$ which is redefined as $\gamma$ in this paper.
	
	
	\begin{figure}[htbp]
		\centerline{\includegraphics[width=120mm,scale=0.9]{FILPD_H2G.png}}
		\caption{An example of applying Gaussian filtering when $\gamma=5$.}
		\label{fig:GAUSSIAN}
	\end{figure}


	\subsubsection*{Linear Interpolation}
	
	The sampling rate for CSI from ESP32 is originally unpredictable and not constant but it is running around 70Hz. So, we do a process called ``Linear Interpolation" to the CSI data in order to maintain the dimension of each sequence. The CSI data can be resampled into any size so, we determined it as 60 Hz.
	
	An example of CSI Linear Interpolation is shown in Fig.~\ref{fig:LINEARINTER} where both look identical but the top is logged at $70-80$ Hz unstably while the bottom is at $60$ Hz stably.
	
To recreate a CSI data at rate 60Hz, we calculate each with 2 data at the closet timestamps from the original with a simple mathematical weight equation as in Eq.~\ref{eq:LINEARINTER} in order to recreate a CSI data with new frequency for all sequences.
	
	\begin{equation}
		\begin{aligned}
			& CSI_{now} = CSI_{before} \\ 
			& + \left(  \frac{ts_{now}-ts_{before}}{ts_{after}-ts_{before}}  \times (CSI_{after}-CSI_{before})   \right),
			\label{eq:LINEARINTER}
		\end{aligned}
	\end{equation}
	where $ts_{now}$, $ts_{before}$, $ts_{after}$ are desired timestamp, timestamp at the closest CSI before the desired timestamp and timestamp at the closest CSI after the desired timestamp respectively. And, $CSI_{now}$, $CSI_{before}$, $CSI_{after}$ are CSI at the desired timestamp, CSI before the desired timestamp and CSI after the desired timestamp respectively.
	
	\begin{figure}[htbp]
		\centerline{\includegraphics[width=120mm,scale=0.9]{FILPD_G2L.png}}
		\caption{An example of applying CSI linear interpolation.}
		\label{fig:LINEARINTER}
	\end{figure}
	
	

		\subsubsection*{Butterworth Low Pass Filtering}
			
			 After all those filtering, the CSI can implicitly show a wave at respiration frequency. Meanwhile, there is also many high frequency wave attached. Since we assume that nothing is moving in sleeping environment other than a sleeping person taking breath, the wave with frequency higher than normal person respiratory rate is discardable.
			 
			 So, we applied Butterworth low pass filtering that can remove the wave with frequency higher than a certain threshold. By using the  frequency slightly higher than common human respiratory rate, we can obtain only the wave of a human breathing as shown in Fig.~\ref{fig:BPL} where the top is an input that is containing high frequency wave and the bottom is an output without frequency wave higher than common human respiratory rate.
			 
			 Parameters of the process are the order of Butterworth filtering and cutoff frequency which are redefined as $\delta$ and $\epsilon$ respectively in this paper.
			
			
			
			\begin{figure}[htbp]
			\centerline{\includegraphics[width=120mm,scale=0.9]{FILPD_L2B.png}}
			\caption{An example of applying Butterworth low pass filtering when $\delta=2$ and $\epsilon=30$.}
			\label{fig:BPL}
				\end{figure}

		
	\subsection*{Respiratory Rate Extraction}
		
		The philosophy we used to extract RR from both data of CSI from ESP32 and pressure from the ground truth is ``First to hit the mean'' as (ref) have proven its reliability.
		
		The examples of applying the technique with different breathing pattern are shown in Fig.~\ref{fig:RREXT_NORMAL}, Fig.~\ref{fig:RREXT_FAST}, Fig.~\ref{fig:RREXT_HOLD} and  Fig.~\ref{fig:RREXT_SLOW} where each orange X are counted as a breath.
		
		\begin{figure}[htbp]
			\centerline{\includegraphics[width=120mm,scale=0.9]{PD2GT_NORMAL02.png}}
			\caption{An example of Respiratory Rate Extraction from prediction (top) and ground truth (bottom) for normal-rate breathing (prediction BPM = 15, ground truth BPM = 16).}
			\label{fig:RREXT_NORMAL}
		\end{figure}
	\begin{figure}[htbp]
		\centerline{\includegraphics[width=120mm,scale=0.9]{PD2GT_FAST02.png}}
		\caption{An example of Respiratory Rate Extraction from prediction (top) and ground truth (bottom) for fast-rate breathing (prediction BPM = 36, ground truth BPM = 38).}
		\label{fig:RREXT_FAST}
	\end{figure}
\begin{figure}[htbp]
	\centerline{\includegraphics[width=120mm,scale=0.9]{PD2GT_HOLD02.png}}
	\caption{An example of Respiratory Rate Extraction from prediction (top) and ground truth (bottom) for holding-rate breathing (prediction BPM = 16, ground truth BPM = 15).}
	\label{fig:RREXT_HOLD}
\end{figure}
\begin{figure}[htbp]
	\centerline{\includegraphics[width=120mm,scale=0.9]{PD2GT_SLOW02.png}}
	\caption{An example of Respiratory Rate Extraction from prediction (top) and ground truth (bottom) for slow-rate breathing (prediction BPM = 6, ground truth BPM = 5).}
	\label{fig:RREXT_SLOW}
\end{figure}

As you can see in Fig.~\ref{fig:RREXT_HOLD}, the straight behavior of the algorithm recognized the low pressure fluctuation as a breath which is incorrect. So, we need a further correction to distinguish if the state of the graph is a breath holding.
	
	

		\subsubsection*{Breath missing Detection}
		
		A parameter to check if the Amplitude is too low to be a breath.

		\subsection*{Ground Truth (Respiratory Belt)}
	
	The Respiratory Belt logs pressure data with a unit of kilopascal. As mentioned in \nameref{HUMANSLEEPRR}, The belt is set to log RR at resolution of 4 Hz. By connecting the belt to a microprocessor as shown in Fig.~\ref{fig:VERNIERBELT}, we can set log resolution at any rate. However, based on our experiment, normal human RR can be determined at resolution of 4 Hz as shown in Fig.~\ref{fig:FILGT_R2H2G2L} (top).
	
	As it can be simply seen that the raw data from the belt is clear but not precise, so we still need more filtering on it. 3 filterings, as shown in Fig.~\ref{fig:MAPINGSTEP} (right), are applied. For more precision, we applied more Linear Interpolation to ensure the resolution is exactly 4 Hz, The result are shown in Fig.~\ref{fig:FILGT_R2H2G2L}
	
	Parameters of the Hampel Filtering process is window size and $\sigma$ which is redefined as $\zeta$ and $\eta$ in this paper. A parameter of the Gaussian Filtering process is $\sigma$ which is redefined as $\theta$ in this paper.
	
	
	\begin{figure}[htbp]
		\centerline{\includegraphics[width=120mm,scale=0.9]{FILGT_R2H2G2L.png}}
		\caption{An example of pressure logged from Vernier Respiratory Belt for normal human taking breath. Without filtering (top), with Hampel Filtering (2nd top) when  $\zeta=1$ and $\eta=-100$, with Gaussian Filtering (3rd top) when  $\theta=1$ and with Linear Interpolation (bottom) with 4 Hz.}
		\label{fig:FILGT_R2H2G2L}
	\end{figure}

	
	% Results and Discussion can be combined.
	\section*{Results}
	
	\subsection*{Data Collection}
	
	We recruited 10 volunteers to sleep in between the devices (2 WiFi anntennae) while wearing a Vernier belt.
	
	The whole data collection is  50 hours of sleep which worth 3,000 data for 60 seconds/RR rate.
	
	\subsection*{Experimental Result}
	\label{result}	
	
	An evaluation is achieved   by the algorithm written in Python 3.8. The code is available on Github$\footnote[1]{https://github.com/rtmtree/CSPS}$.
	
	For error metrics, Mean Absolute Difference (MAD) is a decent indicator to tell how much RR from the ESP32 is different to RR from the belt. With windows of 60 s, MAD is computed by the following equation.
	
	\begin{equation}
		MAD =  \frac{\sum_{i=1}^{n}   \lvert N^{gt}_i - N^{pd}_ i \rvert  }{n} ,
		\label{eq:MAD}
	\end{equation}

where $n$ is a total number of dataset, $N^{gt}_i$ is RR obtained from the belt at index $i$ and $N^{pd}_i$ is RR obtained from the ESP32 at index $i$.

	Table~\ref{table:result1}, Table~\ref{table:result2}, Table~\ref{table:result3} and  Table~\ref{table:result4} show the MAD of 4 breathing pattern with different parameter fine-tunings.
	
	\begin{table}[!ht]
		\begin{adjustwidth}{-2.25in}{0in} % Comment out/remove adjustwidth environment if table fits in text column.
			\centering
			\caption{
				{\bf Table of evaluation result when $\alpha=1$.}}
			\begin{tabular}{l|llllll}
				\backslashbox{Breathing}{Result} &MAD  \\[1pt]
				\hline
				Normal &2.0 \\[1pt]
				Fast &6.03 \\[1pt]
				Hold &1.97  \\[1pt]
				Slow &0.39 \\[1pt]
			\end{tabular}
			\label{table:result1}
		\end{adjustwidth}
	\end{table}


	\iffalse
	\begin{table}[!ht]
		\begin{adjustwidth}{-2.25in}{0in} % Comment out/remove adjustwidth environment if table fits in text column.
			\centering
			\caption{
				{\bf Table of evaluation result when $\alpha=1$.}}
			\begin{tabular}{l|llllll}
				\backslashbox{Breathing}{Predicted} &Wake & REM &Light &Deep \\[1pt]
				\hline
				Normal &0.19 & 0.06 & 0.72 & 0.03 \\[1pt]
				Fast &0.03 & 0.83 & 0.14 & 0.00 \\[1pt]
				Hold &0.01 & 0.05 & 0.91 & 0.03 \\[1pt]
				Slow &0.00 & 0.01 & 0.16 & 0.83 \\[1pt]
			\end{tabular}
			\label{table:result0}
		\end{adjustwidth}
	\end{table}

\fi
	
	\section*{Discussion}
	
	\subsection*{Environment Installation}
	
	The Installation of WiFi antennae can affect tremendously in data. We fixed the positions in both training and testing process as shown in Fig.~\ref{fig:SETUP01}.
	2 antennae are connecting to ESP32s and to the PC afterward for the AP antennae for CSI collection. Lastly, the human sleeping on the bed need to wear Fitbit sleep tracker as shown in Fig.~\ref{fig:SETUP02}.
	

	
	
	\subsection*{CSI Extraction Method}
	
	There are many solutions for extracting WiFi CSI from the ESP32 as mentioned in \nameref{ESP32}. This paper picked the solution from ESP32-CSI-Tool$\footnote[1]{https://github.com/StevenMHernandez/ESP32-CSI-Tool}$ since it is considerably well-written and simple to organize. To change the method of extracting WiFi CSI would affect the result significantly.
	
	
	
	\subsection*{Multiple Person Estimation}
	
	The model does a mapping rule from WiFi CSI with a specific dimension to human sleep stage matrix. So, it is able to only detect single person pose in a range. The annotations originally can result all human sleep stage by using the devices for all specimen. If we do training with multiple sleep stage data instead, we do not believe the result will be well since CSI of a person moving is rarely separatable form others.
	
	\iffalse
	\subsection*{Human speed of moving to $\sigma$}
	
	As the Table~\ref{table:PCKsigma15}, Table~\ref{table:PCKsigma20}, Table~\ref{table:PCKsigma30} and  Table~\ref{table:PCKsigma40} show the difference of quality by different value of $\sigma$, the fine-tuning says that $\sigma=15$ result the best. This implies length of $15$ frames can represent a specific pattern of move best for the used dataset. So, the volunteer who recorded movement in the datasent perform a move slower, $\sigma$ should be extended to cover the proper time peroid of the movement pattern.
	\fi
	
	\section*{Conclusion}
	
	\iffalse
	Currently, there are many issues on security occuring in all ranges of human especially in elderlies and those who are not capable to live solely. To solve those, issues on privacy usually comes instead e.g. recording videos for preventing accident in a house, monitoring of people in a room. Many people do not feel very comfortable on these. So, we seek a solution where we can monitor those activity without a camera needed. 
	\fi
	Currently, there are many concern on sleepness in all group of people. Especially, those who works in extraordinary time period are facing problem of having deficient sleepness. They are pleased to pay for a device that can inform if their sleepness is well enough e.g. smartwatch, smart ring and smart mattress. Those options come with additional and uncomfortable since users' bodies need a contact to an external device. So, we aim for proposing a contactless Sleep Monitoring device.
	
	After having a research, we discoverd that a variation in WiFi called WiFi CSI can tell whether the area is having an activity or moving objects. Moreoever, \cite{chowdhuryTZ}, \cite{zouH} and many other works had been done very well on detecting even what kind of activities is happening in the area. We do not believe that this is the limitation of WiFi CSI. In order to solve the above problem and prove if WiFi CSI is precise enough, we tried to overcome this by extracting a deeper information like Sleep Stage Estimation from it. 
	
	We controlled environments and enhanced WiFi antennae stability as much as possible then mapped it to the Sleep Stage annotated by Fitbit technology. The result was very poor since the WiFi CSI is very vague. It penetrates through most of the things. We learned from \cite{bib20} that WiFi CSI value does not matter than its change. The works used Long-short term memory (LSTM), a neural network where focusing on sequence of the data, and obtained a very good result.  
	
	We applied the idea of LSTM instead and obtained a lot better result with more than 80\% of overall accuracy. Then, we adjusted the model to be suitable for our type of data and did fine-tuning for the frame size to be fit the most for normal human speed of moving while sleeping. So, This paper proposed a model of mapping rule that can takes a sequence of 30-second WiFi circumstance as an input and return an according human sleep stage as an output. The work can help people to monitor Sleep Stage of elderlies and those who are in need of sleep concerning. Moreover as a contact is not needed, users can not differentiate thier sleeping with or without the device. Significantly, the WiFi CSI obtained from ESP32 which is affordable and widely reachable so, people can simply apply the method on their own. The result of various fine-tuned environments and parammeters is acceptable and shown in \nameref{result}.
	
	
	\section*{Acknowledgments}
	We thank to Sirindhorn International Institute of Technology for providing technical environment and supportive information.
	
	
	
	\nolinenumbers
	
	% Either type in your references using
	% \begin{thebibliography}{}
		% \bibitem{}
		% Text
		% \end{thebibliography}
	%
	% or
	%
	% Compile your BiBTeX database using our plos2015.bst
	% style file and paste the contents of your .bbl file
	% here. See http://journals.plos.org/plosone/s/latex for 
	% step-by-step instructions.
	% 
	\begin{thebibliography}{10}
		
		\bibitem{bib1}
		Conant GC, Wolfe KH.
		\newblock {{T}urning a hobby into a job: how duplicated genes find new
			functions}.
		\newblock Nat Rev Genet. 2008 Dec;9(12):938--950.
		
		\bibitem{bib2}
		Ohno S.
		\newblock Evolution by gene duplication.
		\newblock London: George Alien \& Unwin Ltd. Berlin, Heidelberg and New York:
		Springer-Verlag.; 1970.
		
		\bibitem{bib3}
		Magwire MM, Bayer F, Webster CL, Cao C, Jiggins FM.
		\newblock {{S}uccessive increases in the resistance of {D}rosophila to viral
			infection through a transposon insertion followed by a {D}uplication}.
		\newblock PLoS Genet. 2011 Oct;7(10):e1002337.
		
		
		
		
		\bibitem{bib4}
		Osokin D.
		\newblock {{R}eal-time 2D Multi-Person Pose Estimation on CPU: Lightweight OpenPose}.
		\newblock arXiv:1811.12004 [cs.CV]. 2018 Nov;18.
		
		\bibitem{bib5}
		Mehta D, Sotnychenko O, Mueller F, Xu W, Sridhar S, Pons-Moll G, Theobalt C.
		\newblock {{S}ingle-Shot Multi-Person 3D Pose Estimation From Monocular RGB}.
		\newblock arXiv:1712.03453 [cs.CV]. 2018 Aug;28.
		
		
		\bibitem{wangF}
		Wang F, Panev S, Ziyi D, Han J, Huang D.
		\newblock {{C}an WiFi Estimate Person Pose?}.
		\newblock arXiv:1904.00277 [cs.CV]. 2019 Apr;2.
		
		
		
		\bibitem{liuJ}
		Liu J, Liu H, Chen Y, Wang Y, Wang C.
		\newblock {{W}ireless Sensing for Human Activity: A Survey}.
		\newblock IEEE COMMUNICATIONS SURVEYS \& TUTORIALS, VOL. 22, NO. 3, THIRD QUARTER 2020.
		
		\bibitem{chowdhuryTZ}
		Chowdhury TZ, Leung C, Miao CY.
		\newblock {{W}iHACS: Leveraging WiFi for Human Activity Classification using OFDM Subcarriers’ correlation}.
		\newblock IEEE, GlobalSIP 2017.
		
		
		\bibitem{bib9}
		Guo L, Wang L, Liu J, Zhou W, Lu B.
		\newblock {{H}uAc: Human Activity Recognition Using Crowdsourced WiFi
			Signals and Skeleton Data}.
		\newblock Wireless Communications and Mobile Computing
		Volume 2018, Article ID 6163475.
		
		
		
		\bibitem{bib10}
		Wang F, Feng J, Zhao Y, Xiaobin Zhang, Zhang S.
		\newblock {{J}oint Activity Recognition and Indoor
			Localization with WiFi Fingerprints}.
		\newblock arXiv:1904.04964 [cs.CV]. 2019 Jul;18.
		
		
		\bibitem{bib11}
		Al-qaness MAA, Li F, Ma X, Zhang Y, Liu G.
		\newblock {Device-Free Indoor Activity Recognition System}.
		\newblock Appl. Sci. 2016, 6, 329; doi:10.3390.
		
		\bibitem{bib12}
		Wang W, Liu AX, Shahzad M, Ling K, Lu S.
		\newblock {{D}evice-free Human Activity Recognition Using Commercial WiFi Devices}.
		\newblock  IEEE Journal
		on Selected Areas in Communications. DOI 10.1109/JSAC.2017.2679658.
		
		\bibitem{bib13}
		Zhao T, Li F, Tian P.
		\newblock {{A} Deep-Learning Method for Device Activity Detection in mMTC Under Imperfect CSI Based on Variational-Autoencoder}.
		\newblock IEEE TRANSACTIONS ON VEHICULAR TECHNOLOGY, VOL. 69, NO. 7, JULY 2020.
		
		\bibitem{bib14}
		Liu J, Teng G, Hong F.
		\newblock {{H}uman Activity Sensing with Wireless Signals: A Survey}.
		\newblock Sensors 2020, 20, 1210; doi:10.3390/s20041210.
		
		\bibitem{bib15}
		Yousefi S, Narui H, Dayal S, Ermon S, Valaee S.
		\newblock {{A} Survey on Behaviour Recognition Using WiFi Channel State Information}.
		\newblock arXiv:1708.07129 [cs.AI]. 2017 Aug;23.
		
		\bibitem{bib16}
		Chen Z, Zhang L, Jiang C, Cao Z, Cui W.
		\newblock {{W}iFi CSI Based Passive Human Activity Recognition Using Attention Based BLSTM}.
		\newblock  IEEE TRANSACTIONS ON MOBILE COMPUTING, VOL. 18, NO. 11, 2019 Nov.
		
		\bibitem{bib17}
		Li B, Cui W, Wang W, Zhang L, Chen Z, Wu M.
		\newblock {{T}wo-Stream Convolution Augmented Transformer for Human Activity Recognition}.
		\newblock Association for the Advancement of Artificial
		Intelligence, 2021.
		
		\bibitem{bib18}
		Luo Y, Ren J, Wang Z, Sun W, Pan J, Liu J, Pan J, Lin L.
		\newblock {{L}STM Pose Machines}.
		\newblock arXiv:1712.06316 [cs.CV].
		
		\bibitem{bib19}
		Lee K, Lee I, Lee S.
		\newblock {{P}ropagating LSTM: 3D Pose Estimation based on Joint Interdependency}.
		\newblock Computer Vision – ECCV 2018. ECCV 2018. Lecture Notes in Computer Science, vol 11211. Springer, Cham.
		
		\bibitem{bib20}
		Du X, Vasudevan R, Johnson-Roberson M.
		\newblock {{B}io-LSTM: A Biomechanically Inspired Recurrent Neural Network for 3D Pedestrian Pose and Gait Prediction}.
		\newblock arXiv:1809.03705 [cs.RO]. 2019 Sep;13.
		
		\bibitem{bib21}
		Hossain MRI, Little JJ.
		\newblock {{E}xploiting temporal information for 3D human pose estimation}.
		\newblock arXiv:1711.08585 [cs.CV]. 2018 Sep;12.
		
		\bibitem{bib22}
		Pavllo D, Feichtenhofer C, Grangier D, Auli M.
		\newblock {{3}D human pose estimation in video with temporal convolutions and semi-supervised training}.
		\newblock arXiv:1811.11742 [cs.CV]. 2019 Mar;29.
		
		\bibitem{bib23}
		Chen T, Fang C, Shen X, Zhu Y, Chen Z, Luo J.
		\newblock {{A}natomy-aware 3D Human Pose Estimation with Bone-based Pose Decomposition}.
		\newblock arXiv:2002.10322 [cs.CV]. 2021 Jan;26.
		
		\bibitem{bib24}
		Ruiz AH, Porzi L, Bulo SR, Moreno-Noguer F.
		\newblock {{3}D CNNs on Distance Matrices for Human Action Recognition}.
		\newblock MM ’17, Mountain View, CA, USA. 2017 Oct;23–27.
		
		
		\bibitem{hernandezSM}
		Hernandez SM, Bulut E.
		\newblock {{L}ightweight and Standalone IoT based WiFi Sensing for Active Repositioning and Mobility.}
		\newblock IEEE 21st International Symposium on ``A World of Wireless, Mobile and Multimedia Networks" (WoWMoM), 2020.
		
		\bibitem{atifM}
		Atif M, Muralidharan S, Ko H, Yoo B.
		\newblock {{W}i-ESP—A tool for CSI-based Device-Free Wi-Fi Sensing (DFWS).}
		\newblock Journal of Computational Design and Engineering, 2020, 7(5), 644–656.
		
		\bibitem{zouH}
		Zou H, Zhou Y, Yang J, Jiang H, Xie L, Spanos CJ.
		\newblock {{D}eepSense: Device-free Human Activity Recognition via Autoencoder Long-term Recurrent Convolutional Network.}
		\newblock 2018 IEEE International Conference on Communications (ICC), Kansas City, MO, USA, 2018, pp. 1-6, doi: 10.1109/ICC.2018.8422895.
		
		\bibitem{hochreiterS}
		Hochreiter S, Schmidhuber J.
		\newblock {{L}ONG SHORT-TERM MEMORY.}
		\newblock Neural Computation 9(8):1735-1780, 1997.
		
		
		
		
	\end{thebibliography}
	
	
	
\end{document}

